\documentclass{article}
\title{\bf{Fluid Mechanics}}
\author{Nicholas Malaya} \date{}

\begin{document}
\maketitle

%%%%%%%%%%%%%%%%%%%%%%%%%%%%%%%%%%%%%%%%%%%%%%%%%
%
%        Fluid Dynamics
%
%%%%%%%%%%%%%%%%%%%%%%%%%%%%%%%%%%%%%%%%%%%%%%%%%
\newpage
\section{Fluid Dynamics}
%\begin{equation}
%  \frac{d}{dt}\int_{\Omega(t)} T(x,t) dV = \int_{\Omega} \frac{\partial}{\partial t} T dV + \int_{\partial \Omega} \hat n \cdot w T dA = \int_{\Omega}\frac%{\partial}{\partial t} T + \nabla \cdot w T dV
%\end{equation}
%\newline
%\newline
Friction Velocity
\begin{equation}
  u_\tau = \sqrt{\frac{\tau_w}{\rho}}
\end{equation}
\newline
\newline
The momentum thickness is transverse distance by which the boundary should be displaced to compensate for the reduction in momentum of the flowing fluid on account of boundary layer formation. 
\begin{equation}
  \theta = \int^\infty_0 \frac{u(y)}{u_0}(1-\frac{u(y)}{u_0})dy
\end{equation}
\newline
\newline
The reynolds stress tensor is the stress in a fluid due to the random turbulent fluctuations in the fluid momentum:
\begin{equation}
  \rho \overline{u'_i u'_j}
\end{equation}
\newline
\newline
Drag equation: 
\begin{equation}
  F_D = \frac{1}{2}\rho v^2 C_D A
\end{equation}
\newline
\newline
Bernoulli 
\begin{equation}
  \frac{v^2}{2}+gz+\frac{p}{\rho}=constant
\end{equation}
\newline
Assume: steady state, no friction, constant density, points are on streamlines
\newline
\newline
We advance the non-linear terms explicitly. The linear terms
implicitly. The linear terms are the viscous operator. 

%%%%%%%%%%%%%%%%%%%%%%%%%%%%%%%%%%%%%%%%%%%%%%%%%
%
%        Dimensionless Numbers
%
%%%%%%%%%%%%%%%%%%%%%%%%%%%%%%%%%%%%%%%%%%%%%%%%%
\newpage
\section{Dimensionless Numbers}
Schmidt Number:
\begin{equation}
  Sc = \frac{\nu}{D}
\end{equation}
The ratio of momentum diffusivity (viscosity) and mass diffusivity
\newline
\newline
Damkohler Number
\begin{equation}
  Da = \frac{\textrm{fluid time}}{\textrm{chemical reaction time}}
\end{equation}
\newline
\newline
Knudson Number
\begin{equation}
  Kn = \frac{\lambda}{L}
\end{equation}

\newpage
Amat victoria curam. 

\end{document}
