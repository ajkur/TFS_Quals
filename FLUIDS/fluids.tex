\documentclass{article}
\title{\bf{Fluid Mechanics}}
\author{Nicholas Malaya} \date{}

\begin{document}
\maketitle


%%%%%%%%%%%%%%%%%%%%%%%%%%%%%%%%%%%%%%%%%%%%%%%%%
%
%        Introduction
%
%%%%%%%%%%%%%%%%%%%%%%%%%%%%%%%%%%%%%%%%%%%%%%%%%
\newpage
\section{Introduction}

Shear stress
\begin{equation}
    \tau = \mu\frac{du}{dy}
\end{equation}

%%%%%%%%%%%%%%%%%%%%%%%%%%%%%%%%%%%%%%%%%%%%%%%%%
%
%        Fluid Statics
%
%%%%%%%%%%%%%%%%%%%%%%%%%%%%%%%%%%%%%%%%%%%%%%%%%
\section{Fluid Statics}

Incompressible fluid at rest
\begin{equation}
    p_1-p_2 = \gamma (z_2-z_1)
\end{equation}
\newline
\newline
Pressure head
\begin{equation}
    h = \frac{p_1-p_2}{\gamma}
\end{equation}
\newline
\newline
Buoyant force ($V$ is volume)
\begin{equation}
    F_B = \gamma V
\end{equation}

%%%%%%%%%%%%%%%%%%%%%%%%%%%%%%%%%%%%%%%%%%%%%%%%%
%
%        Elementary Fluid Dynamics
%
%%%%%%%%%%%%%%%%%%%%%%%%%%%%%%%%%%%%%%%%%%%%%%%%%
\newpage
\section{Elementary Fluid Dynamics}

Streamlines - lines that are tangent to the velocity vectors throughout the flow field
\newline
\newline
Bernoulli (for assumptions remember ISIS: inviscid, steady, incompressible, on a streamline)
\begin{equation}
  \frac{v^2}{2}+gz+\frac{p}{\rho}=constant
\end{equation}
We advance the non-linear terms explicitly. The linear terms
implicitly. The linear terms are the viscous operator. 
\newline
\newline
Across the streamline
\begin{equation}
    p + \rho\int\frac{V^2}{R}dn + \gamma z = constant
\end{equation}
\newline
\newline
Static pressure - actual thermodynamic pressure of the fluid as it flows, measured by pressure tap on a flat surface
\newline
\newline
Dynamic pressure - $\rho V^2/2$, measured by inserting a tap pointing upstream
\newline
\newline
Stagnation pressure - static pressure plus dynamic pressure
\newline
\newline
Total pressure - Bernoulli equation, static plus dynamic plus hydrostatic.
\newline
\newline
Flow through a flow meter Eq 3.20
\begin{equation}
    Q = A_2\sqrt\frac{2(p_1-p_2)}{\rho[1-(A_2/A_1)^2]}
\end{equation}


%%%%%%%%%%%%%%%%%%%%%%%%%%%%%%%%%%%%%%%%%%%%%%%%%
%
%        Fluid Kinematics
%
%%%%%%%%%%%%%%%%%%%%%%%%%%%%%%%%%%%%%%%%%%%%%%%%%
\newpage
\section{Fluid Kinematics}

2D streamline Eq 4.1, p. 156
Material derivative and acceleration Eq 4.5, 4.3
Convective acceleration p 163
Reynolds Transport Theorem p 170-177



%%%%%%%%%%%%%%%%%%%%%%%%%%%%%%%%%%%%%%%%%%%%%%%%%
%
%        Finite Control Volume Analysis
%
%%%%%%%%%%%%%%%%%%%%%%%%%%%%%%%%%%%%%%%%%%%%%%%%%
\newpage
\section{Finite Control Volume Analysis}

Continuity Integral form Eq 5.5
Moving CV continuity Eq 5.14, 5.16
Linear momentum Eq 5.22
Linear momentum moving cv Eq 5.29
Momentum of momentum Eq 5.42
Energy integral form Eq 5.64
Energy w/ head loss Eq 5.89


%%%%%%%%%%%%%%%%%%%%%%%%%%%%%%%%%%%%%%%%%%%%%%%%%
%
%        Differential Analysis of Fluid Flow
%
%%%%%%%%%%%%%%%%%%%%%%%%%%%%%%%%%%%%%%%%%%%%%%%%%
\newpage
\section{Differential Analysis of Fluid Flow}



%%%%%%%%%%%%%%%%%%%%%%%%%%%%%%%%%%%%%%%%%%%%%%%%%
%
%        Similitude, Dimensional Analysis, and Modeling
%
%%%%%%%%%%%%%%%%%%%%%%%%%%%%%%%%%%%%%%%%%%%%%%%%%
\newpage
\section{Similitude, Dimensional Analysis, and Modeling}



%%%%%%%%%%%%%%%%%%%%%%%%%%%%%%%%%%%%%%%%%%%%%%%%%
%
%        Viscous Flow in Pipes
%
%%%%%%%%%%%%%%%%%%%%%%%%%%%%%%%%%%%%%%%%%%%%%%%%%
\newpage
\section{Viscous Flow in Pipes}

Friction Velocity (ch. 8 p 425)
\begin{equation}
  u_\tau = \sqrt{\frac{\tau_w}{\rho}}
\end{equation}
\newline
\newline
The reynolds stress tensor is the stress in a fluid due to the random turbulent fluctuations in the fluid momentum (ch. 8):
\begin{equation}
  \rho \overline{u'_i u'_j}
\end{equation}


%%%%%%%%%%%%%%%%%%%%%%%%%%%%%%%%%%%%%%%%%%%%%%%%%
%
%        Flow Over Immersed Bodies
%
%%%%%%%%%%%%%%%%%%%%%%%%%%%%%%%%%%%%%%%%%%%%%%%%%
\newpage
\section{Flow Over Immersed Bodies}

Drag equation (ch 9): 
\begin{equation}
  F_D = \frac{1}{2}\rho v^2 C_D A
\end{equation}
\newline
\newline
The momentum thickness is transverse distance by which the boundary should be displaced to compensate for the reduction in momentum of the flowing fluid on account of boundary layer formation. (ch. 9)
\begin{equation}
  \theta = \int^\infty_0 \frac{u(y)}{u_0}(1-\frac{u(y)}{u_0})dy
\end{equation}
\newline
\newline


%%%%%%%%%%%%%%%%%%%%%%%%%%%%%%%%%%%%%%%%%%%%%%%%%
%
%        Compressible Flow
%
%%%%%%%%%%%%%%%%%%%%%%%%%%%%%%%%%%%%%%%%%%%%%%%%%
\newpage
\section{Compressible Flow}


%\begin{equation}
%  \frac{d}{dt}\int_{\Omega(t)} T(x,t) dV = \int_{\Omega} \frac{\partial}{\partial t} T dV + \int_{\partial \Omega} \hat n \cdot w T dA = \int_{\Omega}\frac%{\partial}{\partial t} T + \nabla \cdot w T dV
%\end{equation}
%\newline
%\newline

%%%%%%%%%%%%%%%%%%%%%%%%%%%%%%%%%%%%%%%%%%%%%%%%%
%
%        Dimensionless Numbers
%
%%%%%%%%%%%%%%%%%%%%%%%%%%%%%%%%%%%%%%%%%%%%%%%%%
\newpage
\section{Dimensionless Numbers}
Schmidt Number:
\begin{equation}
  Sc = \frac{\nu}{D}
\end{equation}
The ratio of momentum diffusivity (viscosity) and mass diffusivity
\newline
\newline
Damkohler Number
\begin{equation}
  Da = \frac{\textrm{fluid time}}{\textrm{chemical reaction time}}
\end{equation}
\newline
\newline
Knudson Number
\begin{equation}
  Kn = \frac{\lambda}{L}
\end{equation}

\newpage
Amat victoria curam. 

\end{document}
