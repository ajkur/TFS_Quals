\documentclass{article}
\usepackage{ulem}
\title{\bf{Fluid Mechanics}}
\author{Nicholas Malaya} \date{}

\begin{document}
\maketitle


%%%%%%%%%%%%%%%%%%%%%%%%%%%%%%%%%%%%%%%%%%%%%%%%%
%
%        Introduction
%
%%%%%%%%%%%%%%%%%%%%%%%%%%%%%%%%%%%%%%%%%%%%%%%%%
\newpage
\section{Introduction}

Shear stress
\begin{equation}
    \tau = \mu\frac{du}{dy}
\end{equation}

%%%%%%%%%%%%%%%%%%%%%%%%%%%%%%%%%%%%%%%%%%%%%%%%%
%
%        Fluid Statics
%
%%%%%%%%%%%%%%%%%%%%%%%%%%%%%%%%%%%%%%%%%%%%%%%%%
\section{Fluid Statics}

Incompressible fluid at rest
\begin{equation}
    p_1-p_2 = \gamma (z_2-z_1)
\end{equation}
\newline
\newline
Pressure head
\begin{equation}
    h = \frac{p_1-p_2}{\gamma}
\end{equation}
\newline
\newline
Buoyant force ($V$ is volume)
\begin{equation}
    F_B = \gamma V
\end{equation}

%%%%%%%%%%%%%%%%%%%%%%%%%%%%%%%%%%%%%%%%%%%%%%%%%
%
%        Elementary Fluid Dynamics
%
%%%%%%%%%%%%%%%%%%%%%%%%%%%%%%%%%%%%%%%%%%%%%%%%%
\newpage
\section{Elementary Fluid Dynamics}

Streamlines - lines that are tangent to the velocity vectors throughout the flow field
\newline
\newline
Bernoulli (for assumptions remember ISIS: inviscid, steady, incompressible, on a streamline)
\begin{equation}
  \frac{v^2}{2}+gz+\frac{p}{\rho}=constant
\end{equation}
We advance the non-linear terms explicitly. The linear terms
implicitly. The linear terms are the viscous operator. 
\newline
\newline
Across the streamline
\begin{equation}
    p + \rho\int\frac{V^2}{R}dn + \gamma z = constant
\end{equation}
\newline
\newline
Static pressure - actual thermodynamic pressure of the fluid as it flows, measured by pressure tap on a flat surface
\newline
\newline
Dynamic pressure - $\rho V^2/2$, measured by inserting a tap pointing upstream
\newline
\newline
Stagnation pressure - static pressure plus dynamic pressure
\newline
\newline
Total pressure - Bernoulli equation, static plus dynamic plus hydrostatic.
\newline
\newline
Flow through a flow meter Eq 3.20
\begin{equation}
    Q = A_2\sqrt\frac{2(p_1-p_2)}{\rho[1-(A_2/A_1)^2]}
\end{equation}


%%%%%%%%%%%%%%%%%%%%%%%%%%%%%%%%%%%%%%%%%%%%%%%%%
%
%        Fluid Kinematics
%
%%%%%%%%%%%%%%%%%%%%%%%%%%%%%%%%%%%%%%%%%%%%%%%%%
\newpage
\section{Fluid Kinematics}

2D streamline 
\begin{equation}
    \frac{dy}{dx}=\frac{v}{u}
\end{equation}
\newline
\newline
Material derivative
\begin{equation}
    \frac{D()}{Dt} = \frac{\partial()}{\partial t} + u\frac{\partial()}{\partial x} + v\frac{\partial()}{\partial y} + w\frac{\partial()}{\partial z}
\end{equation}
\newline
\newline
Acceleration (the portion represented by the spatial derivatives is the convective acceleration)
\begin{equation}
    \mathbf{a} = \frac{\partial\mathbf{V}}{\partial t} + u\frac{\partial\mathbf{V}}{\partial x} + v\frac{\partial\mathbf{V}}{\partial y} + w\frac{\partial\mathbf{V}}{\partial z}
\end{equation}
\newline
\newline
Reynolds Transport Theorem (p 170-177) ($V$ is volume)
\begin{equation}
    B=mb
\end{equation}
\begin{equation}
    \frac{dB_{cv}}{dt}=\frac{d}{dt}\left(\int_{cv}\rho b dV\right)
\end{equation}
    

%%%%%%%%%%%%%%%%%%%%%%%%%%%%%%%%%%%%%%%%%%%%%%%%%
%
%        Finite Control Volume Analysis
%
%%%%%%%%%%%%%%%%%%%%%%%%%%%%%%%%%%%%%%%%%%%%%%%%%
\newpage
\section{Finite Control Volume Analysis}

Continuity Integral Form (for a deforming CV the time rate of change term is usually nonzero)
\begin{equation}
    \frac{\partial}{\partial t}\int_{cv}\rho d\sout{V} + \int_{cv}\rho \mathbf{V}\cdot \mathbf{\hat{n}} dA = 0
\end{equation}
\newline
\newline
Moving CV continuity
\begin{equation}
    \mathbf{V} = \mathbf{W} + \mathbf{V}_{cv}
\end{equation}
\noindent where $W$ is the relative velocity, $V_{cv}$ is the velocity of the control volume, and $V$ is the absolute velocity of the fluid.
\begin{equation}
     \frac{\partial}{\partial t}\int_{cv}\rho d\sout{V} + \int_{cv}\rho \mathbf{W}\cdot \mathbf{\hat{n}} dA = 0
\end{equation}
\newline
\newline
Linear momentum 
\begin{equation}
    \frac{\partial}{\partial t}\int_{cv}\mathbf{V}\rho d\sout{V} + \int_{cv}\mathbf{V}\rho \mathbf{V}\cdot \mathbf{\hat{n}} dA = \sum\mathbf{F}
\end{equation}
\newline
\newline
Linear momentum for a moving CV
\begin{equation}
    \int_{cs}\mathbf{W}\rho\mathbf{W}\cdot \mathbf{\hat{n}} dA = \sum\mathbf{F}
\end{equation}
\newline
\newline
Momentum of momentum 
\begin{equation}
    \frac{\partial}{\partial t}\int_{cv}(\mathbf{r}\times\mathbf{V})\rho d\sout{V} + \int_{cv}(\mathbf{r}\times\mathbf{V})\rho \mathbf{V}\cdot \mathbf{\hat{n}} dA = \sum(\mathbf{r}\times\mathbf{F})
\end{equation}
\newline
\newline
% Energy integral form Eq 5.64
% \begin{equation}
    
% \end{equation}
% \newline
% \newline
Energy w/ head loss Eq 5.89
\begin{equation}
    \frac{p_o}{\gamma} + \frac{\alpha_o\bar{V}^2_o}{2g}+z_o = \frac{p_i}{\gamma} + \frac{\alpha_i\bar{V}^2_i}{2g}+z_i + \frac{W_{shaft,in}}{g} - h_L
\end{equation}

%%%%%%%%%%%%%%%%%%%%%%%%%%%%%%%%%%%%%%%%%%%%%%%%%
%
%        Differential Analysis of Fluid Flow
%
%%%%%%%%%%%%%%%%%%%%%%%%%%%%%%%%%%%%%%%%%%%%%%%%%
\newpage
\section{Differential Analysis of Fluid Flow}

Hydraulic diameter
\begin{equation}
  D_h = \frac{4A_c}{P}
\end{equation}

Conservation of Mass
\begin{equation}
  \frac{\partial\rho}{\partial t} + \frac{\partial\rho u}{\partial x} +\frac{\partial\rho v}{\partial y} +\frac{\partial\rho w}{\partial z} = 0
\end{equation}

Streamfunction
\begin{equation}
  u=\frac{\partial\psi}{\partial y}\quad v=-\frac{\partial\psi}{\partial x}
\end{equation}

%%%%%%%%%%%%%%%%%%%%%%%%%%%%%%%%%%%%%%%%%%%%%%%%%
%
%        Similitude, Dimensional Analysis, and Modeling
%
%%%%%%%%%%%%%%%%%%%%%%%%%%%%%%%%%%%%%%%%%%%%%%%%%
\newpage
\section{Similitude, Dimensional Analysis, and Modeling}

% Buckingham Pi Theorem (p. 349, 5th ed)


%%%%%%%%%%%%%%%%%%%%%%%%%%%%%%%%%%%%%%%%%%%%%%%%%
%
%        Viscous Flow in Pipes
%
%%%%%%%%%%%%%%%%%%%%%%%%%%%%%%%%%%%%%%%%%%%%%%%%%
\newpage
\section{Viscous Flow in Pipes}

Derive viscous stress as a function of pressure gradient from N-S equations
critical tube $Re_D$ (p. 404, 5th ed.)
Darcy friction factor (in terms of pressure drop and shear stress) (p. 415, 5th ed)
Reynolds stress (p. 423, 5th ed.)

Friction Velocity (ch. 8 p 425)
\begin{equation}
  u_\tau = \sqrt{\frac{\tau_w}{\rho}}
\end{equation}
\newline
\newline
The reynolds stress tensor is the stress in a fluid due to the random turbulent fluctuations in the fluid momentum (ch. 8):
\begin{equation}
  \rho \overline{u'_i u'_j}
\end{equation}


%%%%%%%%%%%%%%%%%%%%%%%%%%%%%%%%%%%%%%%%%%%%%%%%%
%
%        Flow Over Immersed Bodies
%
%%%%%%%%%%%%%%%%%%%%%%%%%%%%%%%%%%%%%%%%%%%%%%%%%
\newpage
\section{Flow Over Immersed Bodies}

% Derive Boundary Layer Equations
% Derive Displacement Thickness
% Derive Momentum Thickness
% boundary layer assumptions (p. 498, 5th ed.)
% laminar boundary layer equations (p. 498, 5th ed.)
% local friction coefficient (eqn 9.31, p. 506, 5th ed.)
% approximate transitional value of Re_x for flow over a flat plate (p. 507, 5th ed.)
% Why does boundary layer separation occur on a cylinder? (pp. 514-515, 5th ed., also see Incropera p. 423-424)
% value of \delta_\text{m} when u / u_{\infty} = 0.99 for the Blasius solution (p. 500, 5th ed.)


Drag equation (ch 9): 
\begin{equation}
  F_D = \frac{1}{2}\rho v^2 C_D A
\end{equation}
\newline
\newline
The momentum thickness is transverse distance by which the boundary should be displaced to compensate for the reduction in momentum of the flowing fluid on account of boundary layer formation. (ch. 9)
\begin{equation}
  \theta = \int^\infty_0 \frac{u(y)}{u_0}(1-\frac{u(y)}{u_0})dy
\end{equation}
\newline
\newline


%%%%%%%%%%%%%%%%%%%%%%%%%%%%%%%%%%%%%%%%%%%%%%%%%
%
%        Compressible Flow
%
%%%%%%%%%%%%%%%%%%%%%%%%%%%%%%%%%%%%%%%%%%%%%%%%%
\newpage
\section{Compressible Flow}

Isentropic Relations

%\begin{equation}
%  \frac{d}{dt}\int_{\Omega(t)} T(x,t) dV = \int_{\Omega} \frac{\partial}{\partial t} T dV + \int_{\partial \Omega} \hat n \cdot w T dA = \int_{\Omega}\frac%{\partial}{\partial t} T + \nabla \cdot w T dV
%\end{equation}
%\newline
%\newline

%%%%%%%%%%%%%%%%%%%%%%%%%%%%%%%%%%%%%%%%%%%%%%%%%
%
%        Dimensionless Numbers
%
%%%%%%%%%%%%%%%%%%%%%%%%%%%%%%%%%%%%%%%%%%%%%%%%%
\newpage
\section{Dimensionless Numbers}
Schmidt Number:
\begin{equation}
  Sc = \frac{\nu}{D}
\end{equation}
The ratio of momentum diffusivity (viscosity) and mass diffusivity
\newline
\newline
Damkohler Number
\begin{equation}
  Da = \frac{\textrm{fluid time}}{\textrm{chemical reaction time}}
\end{equation}
\newline
\newline
Knudson Number
\begin{equation}
  Kn = \frac{\lambda}{L}
\end{equation}

\newpage
Amat victoria curam. 

\end{document}
