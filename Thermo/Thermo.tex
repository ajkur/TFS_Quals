%%% Preamble
\documentclass[paper=letter, fontsize=11pt]{scrartcl}

\usepackage[english]{babel}                                                         % English language/hyphenation
\usepackage[protrusion=true,expansion=true]{microtype}  
\usepackage{amsmath,amsfonts,amsthm} % Math packages
\usepackage[pdftex]{graphicx}   
\usepackage{url}
\usepackage{array}
\usepackage{parskip}

%%% Custom sectioning
\usepackage{sectsty}
\allsectionsfont{\normalfont\scshape}


%%% Custom headers/footers (fancyhdr package)
\usepackage{fancyhdr}
\pagestyle{fancyplain}
\fancyhead{}                                            % No page header
\fancyfoot[L]{}                                         % Empty 
\fancyfoot[C]{}                                         % Empty
\fancyfoot[R]{\thepage}                                 % Pagenumbering
\renewcommand{\headrulewidth}{0pt}          % Remove header underlines
\renewcommand{\footrulewidth}{0pt}              % Remove footer underlines
\setlength{\headheight}{13.6pt}


%%% Equation and float numbering
\numberwithin{equation}{section}        % Equationnumbering: section.eq#
\numberwithin{figure}{section}          % Figurenumbering: section.fig#
\numberwithin{table}{section}               % Tablenumbering: section.tab#


%%% Maketitle metadata
\newcommand{\horrule}[1]{\rule{\linewidth}{#1}}     % Horizontal rule

\title{
        %\vspace{-1in}  
        \usefont{OT1}{bch}{b}{n}
        \normalfont \normalsize \textsc{TFS Quals Compendium} \\ [25pt]
        \horrule{0.5pt} \\[0.4cm]
        \huge Thermodynamics \\
        \horrule{2pt} \\[0.5cm]
}
\author{
        \normalfont                                 \normalsize
        Andrew Kurzawski\\[-3pt]      \normalsize
        \today
}
\date{}


%%% Begin document
\begin{document}
\maketitle


\section{Introductory Concepts and Definitions}
    
Extensive property - a property who's value for an overall system is the sum of its values for the parts into which the system is divided.
\newline
\newline
Intensive property - a property who's value is independent of the size or extent of a system and may vary from place to place.
\newline
\newline
Property - a quantity is a property if and only if its change in value between two states is independent of the process.
\newline
\newline
Molar and Mass basis
\begin{equation}
    R = \frac{\bar R}{M}
\end{equation}


\newpage
\section{Energy and the First Law of Thermodynamics}

Sign convention for work:
\begin{equation}
W > 0:\text{work done \textit{by} the system}
\end{equation}
\begin{equation}
W < 0:\text{work done \textit{on} the system}
\end{equation}
\newline
\newline
Inexact versus exact differentials: exact differentials do not depend on the process linking two states, inexact differentials do. (Ex: work is inexact and volume is exact.)
\newline
\newline
Sign convention for heat transfer:
\begin{equation}
Q > 0:\text{heat transfer \textit{to} the system}
\end{equation}
\begin{equation}
Q < 0:\text{heat transfer \textit{from} the system}
\end{equation}
\newline
\newline
First law:
\begin{equation}
    \Delta \text{KE} + \Delta \text{PE} + \Delta U = Q - W
\end{equation}
\newline
\newline
Time rate first law:
\begin{equation}
    \frac{d\text{KE}}{dt} + \frac{d\text{PE}}{dt} + \frac{dU}{dt} = \dot Q - \dot W
\end{equation}
\newline
\newline
Thermal efficiency for power cycles:
\begin{equation}
\eta = \frac{W_{cycle}}{Q_{in}} = \frac{Q_{in}-Q_{out}}{Q_{in}}
\end{equation}
\newline
\newline
Coefficient of performance for refrigeration and heat pump cycles:
\begin{equation}
\beta = \frac{Q_{in}}{W_{cycle}} = \frac{Q_{in}}{Q_{out}-Q_{in}}\quad\text{(refrigeration cycle)}
\end{equation}
\begin{equation}
\gamma = \frac{Q_{out}}{W_{cycle}} = \frac{Q_{out}}{Q_{out}-Q_{in}}\quad\text{(heat pump cycle)}
\end{equation}
\newline
\newline
Power:
\begin{equation}
    \dot W = \mathbf{F}\cdot\mathbf{V},\quad\dot W = \mathbf{T}\omega
\end{equation}


\newpage
\section{Evaluating Properties}

Phase (p-T) diagram, identify solid, liquid, gas areas and critical and triple points (p. 84, 6th ed.)
\newline
\newline
p-v diagram, be able to draw isotherms with $T>T_c$, $T=T_c$, $T<T_c$, label two-phase region, critical point, triple line (p. 84, 6th ed.)
\newline
\newline
How does water differ from most substances? Expands on solidification
\newline
\newline
Fluid quality 
\begin{equation}
    x = \frac{m_{vap}}{m_{liq}+m_{vap}}
\end{equation}
\newline
\newline
Label vaporization, condensation, melting, sublimation, and freezing on phase diagram (p. 88, 6th ed.)
\newline
\newline
Definition (and justification of the use of) enthalpy: it is convenient for control volume analysis
\begin{equation}
    H=U+pV
\end{equation}
\newline
\newline
Definition of $c_v$ and $c_p$ 
\begin{equation}
    c_v = \left(\frac{\partial u}{\partial T}\right)_v
\end{equation}
\begin{equation}
    c_p = \left(\frac{\partial h}{\partial T}\right)_p
\end{equation}
\begin{equation}
    k = \frac{c_p}{c_v}
\end{equation}
\newline
\newline
Approximating liquid state properties from two-phase tables
\begin{equation}
    v(T,p)\approx v_f(T)
\end{equation}
\begin{equation}
    u(T,p)\approx u_f(T)
\end{equation}
\begin{equation}
    h(T,p)\approx h_f(T) + v_f(T)[p-p_{sat}(T)]
\end{equation}
\begin{equation}
    s(T,p)\approx s_f(T)
\end{equation}
\newline
\newline
Incompressible substance: $c_p=c_v$
\newline
\newline
Monatomic substance: $c_p=\frac{5}{2}R$
\newline
\newline
Compressibility Factor
\begin{equation}
    Z = \frac{pv}{RT}
\end{equation}
\newline
\newline
Reduced Temperature and Pressure:
\begin{equation}
    T_R = T/T_c,\quad p_R = p/p_c
\end{equation}
where $Z=f(T_R,p_R)$
\newline
\newline
Ideal Gas
\begin{equation}
    c_p(T) = c_v(T)+R
\end{equation}
\begin{equation}
    k = \frac{c_p(T)}{c_v(T)}
\end{equation}
\newline
\newline
Perfect gas is usually interchangeable with ideal gas but can also be considered such that $c_p$ and $c_v$ are not functions of temperature.
\newline
\newline
Polytropic process: $pV^n=constant$
\newline
\newline
Problem 3.10, p. 134 (6th ed.)


\newpage
\section{Control Volume Analysis Using Energy}

Mass rate balance
\begin{equation}
    \frac{dm_{cv}}{dt} = \sum_i \dot m_i - \sum_e \dot m_e
\end{equation}
\newline
\newline
Energy rate balance
\begin{equation}
    \frac{dE_{cv}}{dt} = \dot Q_{cv} -  \dot W_{cv} + \sum_i \dot m_i \left(h_i + \frac{V_i^2}{2} + gz_i \right) - \sum_e \dot m_e \left(h_e + \frac{V_e^2}{2} + gz_e \right) 
\end{equation}
\newline
\newline
Nozzels and Diffusers (plus throttling devices):
\begin{equation}
    0 = h_1 - h_2 + \frac{V_1^2 - V_2^2}{2}\quad\text{(adiabatic)}
\end{equation}
\begin{equation}
    h_1 = h_2\quad\text{neglecting kinetic energy (throttling devices)}
\end{equation}


\newpage
\section{The Second Law of Thermodynamics}

Irreversible process: system and all parts of its surroundings cannot be exactly restored to their respective initial states.
\newline
\newline
Reversible process: system and surrounding scan be returned to their initial states.
\newline
\newline
Some irreversibilities:

1. Heat transfer through a finite temperature differance

2. Unrestrained expansion of a gas or liquid to a lower pressure

3. Spontaneous chemical reaction

4. Spontaneous mixing of matter at different compositions or states

5. Friction

6. Electric current flow through resistance

7. Magnetization or polarization with hysteresis

8. Inelastic deformation
\newline
\newline
Carnot cycle:

Process 1-2: Adiabatic compression

Process 2-3: Isothermal expansion due to heat transfer from reservoir at $T_H$

Process 3-4: Adiabatic expansion

Process 4-1: Isothermal compression due to heat transfer to reservoir at $T_C$
\newline
\newline
Carnot efficiency
\begin{equation}
    \eta = 1 - \frac{T_C}{T_H}
\end{equation}


\newpage
\section{Using Entropy}

Entropy change
\begin{equation}
    d S = \left(\frac{\delta Q}{T}\right)_\text{int rev}
\end{equation}
\newline
\newline
$TdS$ Equations
\begin{equation}
    T dS = dU + p dV
\end{equation}
\begin{equation}
    T dS = dH - V dp
\end{equation}
\newline
\newline
Entropy change equations
\begin{equation}
    ds = \frac{dh}{T}\quad\text{phase change (pressure is constant)}
\end{equation}
\begin{equation}
    s_2-s_1 = c \text{ln}\frac{T_2}{T_1}\quad\text{incompressible}
\end{equation}
\begin{equation}
    s(T_2,v_2)-s(T_1,v_1) = \int_{T_1}^{T_2}c_v(T)\frac{dT}{T} + R \text{ln}\frac{v_2}{v_1}  \quad\text{ideal gas}
\end{equation}
\begin{equation}
    s(T_2,p_2)-s(T_1,p_1) = \int_{T_1}^{T_2}c_p(T)\frac{dT}{T} - R \text{ln}\frac{p_2}{p_1}  \quad\text{ideal gas}
\end{equation}
\newline
\newline
Closed system
\begin{equation}
    S_2-S_1 = \frac{Q}{T_b}+\sigma
\end{equation}
\newline
\newline
Rate balance
\begin{equation}
    \frac{dS}{dt} = \sum_j \frac{\dot Q_j}{T_j} + \sum_i \dot m_i s_i - \sum_e \dot m_e s_e +\dot\sigma_{cv}
\end{equation}
\newline
\newline
Isentropic pressure and specific volume ratio, r stands for relative (needed to use ideal gas table)
\begin{equation}
    \frac{p_2}{p_1}=\frac{p_{r2}}{p_{r1}}
\end{equation}
\begin{equation}
    \frac{v_2}{v_1}=\frac{v_{r2}}{v_{r1}}
\end{equation}
\begin{equation}
    \frac{T_2}{T_1} = \left(\frac{p_2}{p_1}\right)^{(k-1)/k}
\end{equation}
\newline
\newline
Allowed signs for the entropy production term $\sigma$: 0 for no irreverisbilities and greater than 0 if irreversibilities are present
\newline
\newline
Isentropic efficiencies:
\begin{equation}
    \eta_t = \frac{h_1 - h_2}{h_1 - h_{2s}}\quad\text{Turbine}
\end{equation}
\begin{equation}
    \eta_{nozzle} = \frac{V_2^2}{V_{2s}^2}\quad\text{Nozzle}
\end{equation}
\begin{equation}
    \eta_c = \frac{h_{2s} - h_1}{h_2 - h_1}\quad\text{Compressor/Pump}
\end{equation}
\newline
\newline
Work:
\begin{equation}
    \left(\frac{\dot W_{cv}}{\dot m}\right) = -\int_1^2 vdp
\end{equation}
\newline
\newline
For liquid where specific volume is approximately constant
\begin{equation}
    \left(\frac{\dot W_{cv}}{\dot m}\right) = -v(p_2-p_1)
\end{equation}


\newpage
\section{Exergy Analysis}

Exergy of a system (w/o chemical exergy)
\begin{equation}
    \text{E} = (U-U_0) + p_0(V-V_0) - T_0(S-S_0) + \text{KE} + \text{PE}
\end{equation}
\newline
\newline
Specific Exergy
\begin{equation}
    \text{e} = (u-u_0) + p_0(v-v_0) - T_0(s-s_0) + \text{V}^2/2 + gz
\end{equation}
\newline
\newline
Closed System Exergy Balance
\begin{equation}
    \text{E}_2 - \text{E}_1 = \text{E}_q - \text{E}_w - \text{E}_d
\end{equation}
\begin{equation}
    \text{E}_q = \int_1^2\left(1 - \frac{T_0}{T_b}\right)\delta Q\quad\text{exergy transfer with heat}
\end{equation}
\begin{equation}
    \text{E}_w = [W - p_0(V_2-V_1)]\quad\text{exergy transfer with work}
\end{equation}
\begin{equation}
    \text{E}_d = T_0\sigma\quad\text{exergy destruction}
\end{equation}


\newpage
\section{Vapor Power Systems}

Rankine cycle: The basic cycle consists of a turbine, condenser, pump and boiler.
\newline
\newline
Back work ratio
\begin{equation}
    bwr = \frac{\dot W_p/\dot m}{\dot W_t/\dot m}
\end{equation}
\newline
\newline
Ideal Rankine Cycle:

Process 1-2: Isentropic expansion through a turbine from sat vapor at 1 to condenser pressure

Process 2-3: Isobaric heat transfer from the working fluid to saturated liquid at 3

Process 3-4: Isentropic compression to compressed liquid at 4

Process 4-1: Isobaric heat transfer to the fluid in boiler
\newline
\newline
Improvements: 

Superheat beyond saturated vapor before the turbine. This improves the quality at the turbine outlet.

Reheat between turbine stages. This also increases the steam quality at the turbine exit.

Regeneration: Open feedwater heater bleeds vapor from between turbine stages and mixes it back in to preheat the feedwater going in to the boiler. Closed feedwater heaters preheat without mixing the bled vapor and feedwater. Instead, the bled vapor condenses and is added back in at the condenser or an open feedwater heater. Using multiple heaters helps maintain fluid purity (called deaeration).


\newpage
\section{Gas Power Systems}

Air standard analysis: a fixed amount of air modeled as an ideal gas is the working fluid. Combustion is replaced by heat transfer. Exhaust is replaced by constant volume heat transfer at bottom dead center. (Cold air standard assumes constant specific heats)
\newline
\newline
Otto cycle (thermal efficiency increases as compression ratio increases)

Process 1-2: Isentropic compression from bottom dead center to top dead center

Process 2-3: Constant volume heat transfer at top dead center

Process 3-4: Isentropic expansion (power stroke)

Process 4-1: Constant volume heat rejection at bottom dead center
\newline
\newline
Diesel cycle

Process 1-2: Isentropic compression from bottom dead center to top dead center

Process 2-3: Constant pressure heat transfer (first part of power stroke)

Process 3-4: Isentropic expansion (remainder of power stroke)

Process 4-1: Constant volume heat rejection at bottom dead center
\newline
\newline
Dual cycle (closer approximation to internal combustion engines)

Process 1-2: Isentropic compression from bottom dead center to top dead center

Process 2-3: Constant volume heat transfer at top dead center

Process 3-4: Constant pressure heat transfer (first part of power stroke)

Process 4-5: Isentropic expansion (remainder of power stroke)

Process 5-1: Constant volume heat rejection at bottom dead center
\newline
\newline
Brayton cycle, turbine (p. 462, 6th ed)
Brayton cycle, jet engine (p. 486, 6th ed)
\newline
\newline
Ericsson cycle (p. 497, 6th ed)

Stirling cycle (p. 498, 6th ed)


\newpage
\section{Refrigeration and Heat Pump Systems}

refrigeration cycle (p. 535, p. 241, 6th ed.)


\newpage
\section{Thermodynamic Relations}

Fundamental Thermodynamic Functions Eq 11.37
Equations of state from AK notes


\newpage
\section{Ideal Gas Mixture and Psychrometric Applications}

Relative humidity
Humidity ratio


\newpage
\section{Reacting Mixtures and Combustion}

\section{Chemical and Phase Equilibrium}

%%% End document
\end{document}
