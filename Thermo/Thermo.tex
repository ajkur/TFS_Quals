%%% Preamble
\documentclass[paper=letter, fontsize=11pt]{scrartcl}

\usepackage[english]{babel}                                                         % English language/hyphenation
\usepackage[protrusion=true,expansion=true]{microtype}  
\usepackage{amsmath,amsfonts,amsthm} % Math packages
\usepackage[pdftex]{graphicx}   
\usepackage{url}
\usepackage{array}
\usepackage{parskip}

%%% Custom sectioning
\usepackage{sectsty}
\allsectionsfont{\normalfont\scshape}


%%% Custom headers/footers (fancyhdr package)
\usepackage{fancyhdr}
\pagestyle{fancyplain}
\fancyhead{}                                            % No page header
\fancyfoot[L]{}                                         % Empty 
\fancyfoot[C]{}                                         % Empty
\fancyfoot[R]{\thepage}                                 % Pagenumbering
\renewcommand{\headrulewidth}{0pt}          % Remove header underlines
\renewcommand{\footrulewidth}{0pt}              % Remove footer underlines
\setlength{\headheight}{13.6pt}


%%% Equation and float numbering
\numberwithin{equation}{section}        % Equationnumbering: section.eq#
\numberwithin{figure}{section}          % Figurenumbering: section.fig#
\numberwithin{table}{section}               % Tablenumbering: section.tab#


%%% Maketitle metadata
\newcommand{\horrule}[1]{\rule{\linewidth}{#1}}     % Horizontal rule

\title{
        %\vspace{-1in}  
        \usefont{OT1}{bch}{b}{n}
        \normalfont \normalsize \textsc{TFS Quals Compendium} \\ [25pt]
        \horrule{0.5pt} \\[0.4cm]
        \huge Thermodynamics \\
        \horrule{2pt} \\[0.5cm]
}
\author{
        \normalfont                                 \normalsize
        Andrew Kurzawski\\[-3pt]      \normalsize
        \today
}
\date{}


%%% Begin document
\begin{document}
\maketitle

\section{Introductory Concepts and Definitions}
    
Extensive property - a property who's value for an overall system is the sum of its values for the parts into which the system is divided.

Intensive property - a property who's value is independent of the size or extent of a system and may vary from place to place.

Property - a quantity is a property if and only if its change in value between two states is independent of the process.


\section{Energy and the First Law of Thermodynamics}

Sign convention for work:

\begin{equation}
W > 0:\text{work done {\it by} the system}
\end{equation}

\begin{equation}
W < 0:\text{work done {\it on} the system}
\end{equation}

Inexact versus exact differentials: exact differentials do not depend on the process linking two states, inexact differentials do. (Ex: work is inexact and volume is exact.)

Sign convention for heat transfer:

\begin{equation}
Q > 0:\text{heat transfer {\it to} the system}
\end{equation}

\begin{equation}
Q < 0:\text{heat transfer {\it from} the system}
\end{equation}

First law:

\begin{equation}
\Delta KE + \Delta PE + \Delta U = Q - W
\end{equation}

Thermal efficiency for power cycles:

\begin{equation}
\eta = \frac{W_{cycle}}{Q_{in}} = \frac{Q_{in}-Q_{out}}{Q_{in}}
\end{equation}

Coefficient of performance for refrigeration and heat pump cycles:

\begin{equation}
\beta = \frac{Q_{in}}{W_{cycle}} = \frac{Q_{in}}{Q_{out}-Q_{in}}\quad\text{(refrigeration cycle)}
\end{equation}

\begin{equation}
\gamma = \frac{Q_{out}}{W_{cycle}} = \frac{Q_{out}}{Q_{out}-Q_{in}}\quad\text{(heat pump cycle)}
\end{equation}


\section{Evaluating Properties}

% phase (p-T) diagram, identify solid, liquid, gas areas and critical and triple points (p. 84, 6th ed.)
% p-v diagram, be able to draw isotherms with T>T_c, T=T_c, T<T_c, label two-phase region, critical point, triple line (p. 84, 6th ed.)
% how does water differ from most substances?
% fluid quality (p. 87, 6th ed.)
% label vaporization, condensation, melting, sublimation, and freezing on phase diagram (p. 88, 6th ed.)
% definition (and justification of the use of) enthalpy (p. 95, p. 157, 6th ed.)
% definition of c_v, c_p (p. 105, 6th ed.)
% definition of adiabatic exponent/specific heat ratio (p. 105, 6th ed.)
% approximating liquid state properties from two-phase tables (p. 106, 6th ed.)
% problem 3.10, p. 134 (6th ed.)

% \section{Control Volume Analysis Using Energy}
%     energy rate balance (eqn. 4.15, p. 157, 6th ed.)

% \section{The Second Law of Thermodynamics}
%     definition of irreversible/reversible processes (p. 219, 6th ed.)
%     Carnot cycle (p. 239, p. 266, 6th ed.)

% \section{Using Entropy}
%     d S = \left(\frac{\delta Q}{T}\right)_\text{int rev} (eqn. 6.2b, p. 256, 6th ed.)
%     T dS = dU + p dV (eqn. 6.8, p. 261, 6th ed.)
%     T dS = dH - V dp (eqn. 6.9, p. 261, 6th ed.)
%     entropy change equations
%     entropy rate balance (eqn. 6.36, p. 283, 6th ed.)
%     isentropic pressure and specific volume ratio (needed to use ideal gas table) (pp. 292-293, 6th ed.)
%     differential form of the entropy balance for a closed system (eqn. 6.25, p. 269, 6th ed.)
%     allowed signs for the entropy production term (p. 271, 6th ed.)
%     isentropic efficiencies (pp. 297-303, 6th ed.)

% \section{Exergy Analysis}

% \section{Vapor Power Systems}
%     Rankine cycle (p. 392, 6th ed.)

% \section{Gas Power Systems}
%     Otto cycle (p. 448, 6th ed)
%     Diesel cycle (p. 453, 6th ed)
%     Dual cycle (p. 457, 6th ed)
%     Brayton cycle, turbine (p. 462, 6th ed)
%     Brayton cycle, jet engine (p. 486, 6th ed)
%     Ericsson cycle (p. 497, 6th ed)
%     Stirling cycle (p. 498, 6th ed)

% \section{Refrigeration and Heat Pump Systems}
%     refrigeration cycle (p. 535, p. 241, 6th ed.)

% \section{Thermodynamic Relations}
%     Fundamental Thermodynamic Functions Eq 11.37
%     Equations of state from AK notes

% \section{Ideal Gas Mixture and Psychrometric Applications}

% \section{Reacting Mixtures and Combustion}

% \section{Chemical and Phase Equilibrium}

%%% End document
\end{document}
