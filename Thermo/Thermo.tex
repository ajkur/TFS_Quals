%%% Preamble
\documentclass[paper=letter, fontsize=11pt]{scrartcl}

\usepackage[english]{babel}                                                         % English language/hyphenation
\usepackage[protrusion=true,expansion=true]{microtype}  
\usepackage{amsmath,amsfonts,amsthm} % Math packages
\usepackage[pdftex]{graphicx}   
\usepackage{url}
\usepackage{array}
\usepackage{parskip}

%%% Custom sectioning
\usepackage{sectsty}
\allsectionsfont{\normalfont\scshape}


%%% Custom headers/footers (fancyhdr package)
\usepackage{fancyhdr}
\pagestyle{fancyplain}
\fancyhead{}                                            % No page header
\fancyfoot[L]{}                                         % Empty 
\fancyfoot[C]{}                                         % Empty
\fancyfoot[R]{\thepage}                                 % Pagenumbering
\renewcommand{\headrulewidth}{0pt}          % Remove header underlines
\renewcommand{\footrulewidth}{0pt}              % Remove footer underlines
\setlength{\headheight}{13.6pt}


%%% Equation and float numbering
\numberwithin{equation}{section}        % Equationnumbering: section.eq#
\numberwithin{figure}{section}          % Figurenumbering: section.fig#
\numberwithin{table}{section}               % Tablenumbering: section.tab#


%%% Maketitle metadata
\newcommand{\horrule}[1]{\rule{\linewidth}{#1}}     % Horizontal rule

\title{
        %\vspace{-1in}  
        \usefont{OT1}{bch}{b}{n}
        \normalfont \normalsize \textsc{TFS Quals Compendium} \\ [25pt]
        \horrule{0.5pt} \\[0.4cm]
        \huge Thermodynamics \\
        \horrule{2pt} \\[0.5cm]
}
\author{
        \normalfont                                 \normalsize
        Andrew Kurzawski\\[-3pt]      \normalsize
        \today
}
\date{}


%%% Begin document
\begin{document}
\maketitle


\section{Introductory Concepts and Definitions}
    
Extensive property - a property who's value for an overall system is the sum of its values for the parts into which the system is divided.

Intensive property - a property who's value is independent of the size or extent of a system and may vary from place to place.

Property - a quantity is a property if and only if its change in value between two states is independent of the process.


\newpage
\section{Energy and the First Law of Thermodynamics}

Sign convention for work:

\begin{equation}
W > 0:\text{work done {\it by} the system}
\end{equation}

\begin{equation}
W < 0:\text{work done {\it on} the system}
\end{equation}

Inexact versus exact differentials: exact differentials do not depend on the process linking two states, inexact differentials do. (Ex: work is inexact and volume is exact.)

Sign convention for heat transfer:

\begin{equation}
Q > 0:\text{heat transfer {\it to} the system}
\end{equation}

\begin{equation}
Q < 0:\text{heat transfer {\it from} the system}
\end{equation}

First law:

\begin{equation}
\Delta KE + \Delta PE + \Delta U = Q - W
\end{equation}

Thermal efficiency for power cycles:

\begin{equation}
\eta = \frac{W_{cycle}}{Q_{in}} = \frac{Q_{in}-Q_{out}}{Q_{in}}
\end{equation}

Coefficient of performance for refrigeration and heat pump cycles:

\begin{equation}
\beta = \frac{Q_{in}}{W_{cycle}} = \frac{Q_{in}}{Q_{out}-Q_{in}}\quad\text{(refrigeration cycle)}
\end{equation}

\begin{equation}
\gamma = \frac{Q_{out}}{W_{cycle}} = \frac{Q_{out}}{Q_{out}-Q_{in}}\quad\text{(heat pump cycle)}
\end{equation}


\newpage
\section{Evaluating Properties}

Phase (p-T) diagram, identify solid, liquid, gas areas and critical and triple points (p. 84, 6th ed.)

p-v diagram, be able to draw isotherms with $T>T_c$, $T=T_c$, $T<T_c$, label two-phase region, critical point, triple line (p. 84, 6th ed.)

How does water differ from most substances? Expands on solidification

Fluid quality 
\begin{equation}
    x = \frac{m_{vap}}{m_{liq}+m_{vap}}
\end{equation}

Label vaporization, condensation, melting, sublimation, and freezing on phase diagram (p. 88, 6th ed.)

Definition (and justification of the use of) enthalpy: it is convenient for control volume analysis

\begin{equation}
    H=U+pV
\end{equation}

Definition of $c_v$ and $c_p$ 
\begin{equation}
    c_v = \left(\frac{\partial u}{\partial T}\right)_v
\end{equation}
\begin{equation}
    c_p = \left(\frac{\partial h}{\partial T}\right)_p
\end{equation}
\begin{equation}
    k = \frac{c_p}{c_v}
\end{equation}

Approximating liquid state properties from two-phase tables
\begin{equation}
    v(T,p)\approx v_f(T)
\end{equation}
\begin{equation}
    u(T,p)\approx u_f(T)
\end{equation}
\begin{equation}
    h(T,p)\approx h_f(T) + v_f(T)[p-p_{sat}(T)]
\end{equation}

Incompressible substance: $c_p=c_v$

Monatomic substance: $c_p=\frac{5}{2}R$

Compressibility Factor
\begin{equation}
    Z = \frac{pv}{RT}
\end{equation}

Reduced Temperature and Pressure:
\begin{equation}
    T_R = T/T_c,\quad p_R = p/p_c
\end{equation}
where $Z=f(T_R,p_R)$

Ideal Gas
\begin{equation}
    c_p(T) = c_v(T)+R
\end{equation}
\begin{equation}
    k = \frac{c_p(T)}{c_v(T)}
\end{equation}

Perfect gas is usually interchangeable with ideal gas but can also be considered such that $c_p$ and $c_v$ are not functions of temperature.

Polytropic process: $pV^n=constant$

Problem 3.10, p. 134 (6th ed.)


\newpage
\section{Control Volume Analysis Using Energy}

Mass rate balance
\begin{equation}
    \frac{dm_{cv}}{dt} = \sum_i \dot m_i - \sum_e \dot m_e
\end{equation}

Energy rate balance
\begin{equation}
    \frac{dE_{cv}}{dt} = \dot Q_{cv} -  \dot W_{cv} + \sum_i \dot m_i \left(h_i + \frac{V_i^2}{2} + gz_i \right) - \sum_e \dot m_e \left(h_e + \frac{V_e^2}{2} + gz_e \right) 
\end{equation}

Nozzels and Diffusers (plus throttling devices):
\begin{equation}
    0 = h_1 - h_2 + \frac{V_1^2 - V_2^2}{2}\quad\text{(adiabatic)}
\end{equation}
\begin{equation}
    h_1 = h_2\quad\text{neglecting kinetic energy (throttling devices)}
\end{equation}


\newpage
\section{The Second Law of Thermodynamics}

Irreversible process: system and all parts of its surroundings cannot be exactly restored to their respective initial states.

Reversible process: system and surrounding scan be returned to their initial states.
\newline
\newline
Some irreversibilities:

1. Heat transfer through a finite temperature differance

2. Unrestrained expansion of a gas or liquid to a lower pressure

3. Spontaneous chemical reaction

4. Spontaneous mixing of matter at different compositions or states

5. Friction

6. Electric current flow through resistance

7. Magnetization or polarization with hysteresis

8. Inelastic deformation
\newline
\newline
Carnot cycle:

Process 1-2: Adiabatic compression

Process 2-3: Isothermal expansion due to heat transfer from reservoir at $T_H$

Process 3-4: Adiabatic expansion

Process 4-1: Isothermal compression due to heat transfer to reservoir at $T_C$

Carnot efficiency

\begin{equation}
    \eta = 1 - \frac{T_C}{T_H}
\end{equation}

\section{Using Entropy}

Entropy change
\begin{equation}
    d S = \left(\frac{\delta Q}{T}\right)_\text{int rev}
\end{equation}

$TdS$ Equations
\begin{equation}
    T dS = dU + p dV
\end{equation}
\begin{equation}
    T dS = dH - V dp
\end{equation}

Entropy change equations

\begin{equation}
    s_2-s_1 = c \text{ln}\frac{T_2}{T_1}\quad\text{incompressible}
\end{equation}

\begin{equation}
    s(T_2,v_2)-s(T_1,v_1) = \int_{T_1}^{T_2}c_v(T)\frac{dT}{T} + R \text{ln}\frac{v_2}{v_1}  \quad\text{ideal gas}
\end{equation}

\begin{equation}
    s(T_2,p_2)-s(T_1,p_1) = \int_{T_1}^{T_2}c_p(T)\frac{dT}{T} - R \text{ln}\frac{p_2}{p_1}  \quad\text{ideal gas}
\end{equation}

Closed system

\begin{equation}
    S_2-S_1 = \frac{Q}{T_b}+\sigma
\end{equation}

Rate balance

\begin{equation}
    \frac{dS}{dt} = \sum_j \frac{\dot Q_j}{T_j} + \sum_i \dot m_i s_i - \sum_e \dot m_e s_e +\dot\sigma_{cv}
\end{equation}

Isentropic pressure and specific volume ratio, r stands for relative (needed to use ideal gas table)

\begin{equation}
    \frac{p_2}{p_1}=\frac{p_{r2}}{p_{r1}}
\end{equation}
\begin{equation}
    \frac{v_2}{v_1}=\frac{v_{r2}}{v_{r1}}
\end{equation}
\begin{equation}
    \frac{T_2}{T_1} = \left(\frac{p_2}{p_1}\right)^{(k-1)/k}
\end{equation}

Allowed signs for the entropy production term (p. 271, 6th ed.): 0 for no irreverisbilities and greater than 0 for if irreversibilities are present

Isentropic efficiencies:

\begin{equation}
    \eta_t = \frac{h_1 - h_2}{h_1 - h_{2s}}\quad\text{Turbine}
\end{equation}

\begin{equation}
    \eta_{nozzle} = \frac{V_2^2}{V_{2s}^2}\quad\text{Nozzle}
\end{equation}

\begin{equation}
    \eta_c = \frac{h_{2s} - h_1}{h_2 - h_1}\quad\text{Compressor/Pump}
\end{equation}

% \section{Exergy Analysis}

% \section{Vapor Power Systems}
%     Rankine cycle (p. 392, 6th ed.)

% \section{Gas Power Systems}
%     Otto cycle (p. 448, 6th ed)
%     Diesel cycle (p. 453, 6th ed)
%     Dual cycle (p. 457, 6th ed)
%     Brayton cycle, turbine (p. 462, 6th ed)
%     Brayton cycle, jet engine (p. 486, 6th ed)
%     Ericsson cycle (p. 497, 6th ed)
%     Stirling cycle (p. 498, 6th ed)

% \section{Refrigeration and Heat Pump Systems}
%     refrigeration cycle (p. 535, p. 241, 6th ed.)

% \section{Thermodynamic Relations}
%     Fundamental Thermodynamic Functions Eq 11.37
%     Equations of state from AK notes

% \section{Ideal Gas Mixture and Psychrometric Applications}

% \section{Reacting Mixtures and Combustion}

% \section{Chemical and Phase Equilibrium}

%%% End document
\end{document}
